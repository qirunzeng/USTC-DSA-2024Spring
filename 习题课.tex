\documentclass{beamer}

\usepackage{ctex}
\usepackage{booktabs}




% 选择主题
\usetheme{Madrid}
\usepackage{amsmath, amssymb, amsthm}
\usepackage[ruled, linesnumbered, boxed]{algorithm2e}

\author{Qirun Zeng、Weixin Chen}
\date{\today}
\title{数据结构及其算法习题课}

% \AtBeginSection[]{
%     \begin{frame}
%         \frametitle{目录}
%         \tableofcontents[currentsection]
%     \end{frame}
% }

\AtBeginSubsection[]{
    \begin{frame}
        \frametitle{目录}
        \tableofcontents[currentsection, currentsubsection]
    \end{frame}
}

\begin{document}
    \begin{frame}
        \titlepage
    \end{frame}

    \section{作业}

    \subsection{线性表}

    \begin{frame}
        \frametitle{习题 2.6}
        试写一个算法,对带头结点的单链表实现就地逆置。
    \end{frame}

    \begin{frame}
        \frametitle{习题 2.6}
        试写一个算法,对带头结点的单链表实现就地逆置。
        \begin{block}{分析}
            \begin{itemize}
                \item<1-> 从头结点的下一个结点开始,采用头插法,逐个将后继结点插入到头结点之后
                \item<2-> 逆置完成后,头结点的后继结点即为原链表的尾结点
            \end{itemize}
        \end{block}
    \end{frame}

    \begin{frame}
        \frametitle{习题 2.6}
        试写一个算法,对带头结点的单链表实现就地逆置。
        \begin{algorithm}[H]
            \caption{就地逆置}
            \SetKwFunction{Fmain}{Reverse}
            \SetKwProg{Fn}{Function}{:}{}
            \Fn{\Fmain{$L$}}{
                $p \gets L.next$\;
                $L.next \gets \text{NULL}$\;
                \While{$p \neq \text{NULL}$}{
                    $q \gets p.next$\;
                    $p.next \gets L.next$\;
                    $L.next \gets p$\;
                    $p \gets q$\;
                }
            }
        \end{algorithm}
    \end{frame}

    \subsection{树和二叉树}
    \begin{frame}
        \frametitle{编写算法,将二叉树所有左右子树互换}
        \begin{algorithm}[H]
            \caption{Mirror}
            \SetKwFunction{Fmain}{Mirror}
            \SetKwProg{Fn}{Function}{:}{}
            \Fn{\Fmain{$T$}}{
                \If{$T = \text{NULL}$}{
                    \Return{\text{NULL}}\;
                }
                $T.left, T.right \gets \text{Mirror}(T.right), \text{Mirror}(T.left)$\;
                \Return{$T$}\;
            }
        \end{algorithm}
    \end{frame}

    \begin{frame}
        \frametitle{先序遍历和中序遍历序列构造二叉树}
        \begin{algorithm}[H]
            \caption{PreInCreate}
            \SetKwFunction{Fmain}{PreInCreate}
            \SetKwProg{Fn}{Function}{:}{}
            \Fn{\Fmain{$pre$, $in$, $n$}}{
                \If{$n = 0$}{
                    \Return{\text{NULL}}\;
                }
                $T \gets \text{new Node}$\;
                $T.data \gets pre[0]$\;
                $k \gets 0$\;
                \While{$in[k] \neq pre[0]$}{
                    $k \gets k+1$\;
                }
                $T.left \gets \text{PreInCreate}(pre+1, in, k)$\;
                $T.right \gets \text{PreInCreate}(pre+k+1, in+k+1, n-k-1)$\;
                \Return{$T$}\;
            }
        \end{algorithm}
    \end{frame}

    \begin{frame}
        \frametitle{中序遍历和后序遍历序列构造二叉树}
        \begin{algorithm}[H]
            \caption{InPostCreate}
            \SetKwFunction{Fmain}{InPostCreate}
            \SetKwProg{Fn}{Function}{:}{}
            \Fn{\Fmain{$in$, $post$, $n$}}{
                \If{$n = 0$}{
                    \Return{\text{NULL}}\;
                }
                $T \gets \text{new Node}$\;
                $T.data \gets post[n-1]$\;
                $k \gets 0$\;
                \While{$in[k] \neq post[n-1]$}{
                    $k \gets k+1$\;
                }
                $T.left \gets \text{InPostCreate}(in, post, k)$\;
                $T.right \gets \text{InPostCreate}(in+k+1, post+k, n-k-1)$\;
                \Return{$T$}\;
            }
        \end{algorithm}
    \end{frame}

    \begin{frame}
        \frametitle{判断一棵树是不是完全二叉树}
        \begin{algorithm}[H]
            \caption{IsComplete}
            \SetKwFunction{Fmain}{IsComplete}
            \SetKwProg{Fn}{Function}{:}{}
            \Fn{\Fmain{$T$}}{
                $Q.push(T), flag \gets \text{false}$\;
                \While{$Q \neq \text{empty}$}{
                    $p \gets Q.pop()$\;
                    \If{$p = \text{NULL}$}{
                        $flag \gets \text{true}$\;
                    }
                    \Else{
                        \If{$flag$}{
                            \Return{\text{false}}\;
                        }
                        $Q.push(p.left), Q.push(p.right)$\;
                    }
                }
                \Return{\text{true}}\;
            }
        \end{algorithm}
    \end{frame}

    \begin{frame}
        \frametitle{习题 6.4}
        一棵深度为 $H$ 的满二叉树具有如下性质:第 $H$ 层上所有结点都是叶子结点,其 余各层上每个结点都有 $k$ 棵非空子树。如果从 $1$ 开始按自上而下、自左向右的次序对全部结点编号,问:
        \begin{enumerate}
            \item 各层的结点数目是多少?
            \item 编号为 $i$ 的结点的父结点(若存在)的编号是多少?
            \item 编号为 $i$ 的结点的第 $j$ 个孩子(若存在)的编号是多少?
            \item 编号为 $i$ 的结点有右兄弟的条件是什么?其右兄弟的编号是多少?
        \end{enumerate}
    \end{frame}

    \begin{frame}
        \frametitle{answer}
        \begin{enumerate}
            \item 第 $i$ 层有 $k^{i-1}$ 个结点
            \item $i$ 的父结点编号为 $\left\lfloor \frac{i-2}{k} \right\rfloor + 1$
            \item $i \times k + j - k + 1$
            \item 需要满足:\begin{align}
                i & \neq \left(\left\lfloor \frac{i-2}{k}\right\rfloor+1\right) \cdot k + k - k + 1 \\
                i -\left\lfloor \frac{i-2}{k}\right\rfloor \cdot k & \neq k + 1 \\
                2 + (i-2) \mod k & \neq k + 1 \\
                i \mod k & \neq 1
            \end{align}
        \end{enumerate}
    \end{frame}

    \subsection{查找}
    \begin{frame}
        \frametitle{习题 9.7}
        判断一棵树是不是二叉排序树
        \begin{algorithm}[H]
            \caption{判断二叉排序树}
            \SetKwFunction{Fmain}{IsBST}
            \SetKwProg{Fn}{Function}{:}{}
            \Fn{\Fmain{$T, \text{min}, \text{max}$}}{
                \If{$T = \text{NULL}$}{
                    \Return{\text{true}}\;
                }
                \If{$T.data \leq \text{min}$ or $T.data \geq \text{max}$}{
                    \Return{\text{false}}\;
                }
                \Return{\text{IsBST}(T.left, \text{min}, T.data) and \text{IsBST}(T.right, T.data, \text{max})}\;
            }
        \end{algorithm}
    \end{frame}

    \subsection{内部排序}

    \begin{frame}
        \frametitle{习题 10.3}
        判别以下序列是否为堆(小顶堆或大顶堆),若不是,则把它调整为堆
        \begin{enumerate} 
            \item (96,86,48,73,35,39,42,57,66,21);
            \item (12,70,33,65,24,56,48,92,86,33);
        \end{enumerate}
        遇到堆的题目,建议先转化为完全二叉树,判断是否满足堆的性质
    \end{frame}

    \begin{frame}
        \frametitle{堆的插入}
        把堆想象成一个线性表,直接在线性表的末尾插入新元素,然后调整堆的性质。一个结点的父亲的下标为 $i/2$,左儿子的下标为 $2i$,右儿子的下标为 $2i+1$。
        \begin{algorithm}[H]
            \caption{Insert}
            \SetKwFunction{Fmain}{Insert}
            \SetKwProg{Fn}{Function}{:}{}
            \Fn{\Fmain{$H$, $x$}}{
                $H.data[H.size] \gets x$\;
                $i \gets H.size$\;
                \While{$i > 1$ and $x < H.data[i/2]$}{
                    $H.data[i] \gets H.data[i/2]$\;
                    $i \gets i/2$\;
                }
                $H.data[i] \gets x$\;
                $H.size \gets H.size+1$\;
            }
        \end{algorithm}
    \end{frame}

    \begin{frame}
        \frametitle{堆的删除}
        删除堆顶元素,把堆的最后一个元素放到堆顶,自上而下调整
        \begin{algorithm}[H]
            \caption{Delete}
            \SetKwFunction{Fmain}{Delete}
            \SetKwProg{Fn}{Function}{:}{}
            \Fn{\Fmain{$H$}}{
                $x \gets H.data[1], y \gets H.data[H.size-1], H.size \gets H.size-1, i \gets 1$\;
                \While{$2i < H.size$}{
                    $j \gets 2i$\;
                    \If{$j < H.size$ and $H.data[j] > H.data[j+1]$}{
                        $j \gets j+1$\;
                    }
                    \If{$y \leq H.data[j]$}{
                        \textbf{break}\;
                    }
                    $H.data[i] \gets H.data[j], i \gets j$\;
                }
                $H.data[i] \gets y$\;
                \Return{$x$}\;
            }
        \end{algorithm}
    \end{frame}

    \begin{frame}
        \frametitle{习题 10.5}
        \begin{algorithm}[H]
            \caption{试以单链表为存储结构,实现简单选择排序算法}
            \SetKwFunction{Fmain}{SelectSort}
            \SetKwProg{Fn}{Function}{:}{}
            \Fn{\Fmain{$L$}}{
                $p \gets L.next$\;
                \While{$p \neq \text{NULL}$}{
                    $q \gets p.next$\;
                    $min \gets p$\;
                    \While{$q \neq \text{NULL}$}{
                        \If{$q.data < min.data$}{
                            $min \gets q$\;
                        }
                        $q \gets q.next$\;
                    }
                    $\text{swap}(p.data, min.data)$\;
                    $p \gets p.next$\;
                }
            }
        \end{algorithm}
    \end{frame}

    \section{实验}

    \subsection{约瑟夫环}
    \begin{frame}
        \frametitle{约瑟夫环}
        \begin{itemize}
            \item<1-> \textbf{思路:} 只需要一个循环链表,每次删除第 $m$ 个结点,直到链表为空
            \item<2-> \textbf{实现:} 略
            \item<3-> \textbf{优化:} 考虑到长度只有 $n$,而 $m$ 可能很大,可以通过取模运算优化
            \item<4-> \textbf{bug:} 取模运算的结果可能为 $0$,需要特殊处理
            \item<5-> \textbf{实现:} $m = (m-1)\%(n--)$+1;
        \end{itemize}
    \end{frame}

    \subsection{串的模式匹配算法}
    \begin{frame}
        \frametitle{串的模式匹配算法}
        \begin{algorithm}[H]
            \caption{Next}
            \SetKwFunction{Fmain}{Next}
            \SetKwProg{Fn}{Function}{:}{}
            \Fn{\Fmain{$s$, $t$}}{
                $next[0] \gets -1$\;
                $j \gets 0, k \gets -1$\;
                \While{$j < t.length$}{
                    \If{$k = -1$ or $t[j] = t[k]$}{
                        $j \gets j+1, k \gets k+1$\;
                        $next[j] \gets k$\;
                    }
                    \Else{
                        $k \gets next[k]$\;
                    }
                }
                \Return{$next$}\;
            }
        \end{algorithm}
    \end{frame}
    
    \subsection{最小生成树问题}
    \begin{frame}
        \frametitle{最小生成树问题}
        \begin{itemize}
            \item Kruskal 算法
            \item Prim 算法
        \end{itemize}
    \end{frame}

    \begin{frame}
        \frametitle{Kruskal 算法}
        \begin{algorithm}[H]
            \caption{Kruskal}
            \SetKwFunction{Fmain}{Kruskal}
            \SetKwProg{Fn}{Function}{:}{}
            \Fn{\Fmain{$G$}}{
                $E \gets \text{sort}(G.edges)$\;
                $T \gets \text{NULL}$\;
                \For{$i \gets 1$ \KwTo $G.vexnum$}{
                    $parent[i] \gets i$\;
                }
                \For{$i \gets 1$ \KwTo $E.length$}{
                    \If{$\text{Find}(E[i].u) \neq \text{Find}(E[i].v)$}{
                        $T \gets T \cup \{E[i]\}$\;
                        $\text{Union}(E[i].u, E[i].v)$\;
                    }
                }
                \Return{$T$}\;
            }
        \end{algorithm}
    \end{frame}
    
    \section{复习}

    \subsection{图}
    \begin{frame}
        \frametitle{图}
        \begin{block}{基本概念}
            \begin{itemize}
                \item<1-> 逻辑结构相关:
                \begin{itemize}
                    \item<2-> 顶点,边,度,路径,环,有向/无向图,图/网,生成树,搜索树
                    \item<3-> 连通/强连通,拓扑排序,关键路径,最短路径,最小生成树
                \end{itemize}
                \item<4-> 存储结构相关:
                \begin{itemize}
                    \item<5-> 十字链表,邻接多重表,边表:给一个图,能写出各种存储结构
                    \item<6-> 顶点数组+邻接表/邻接矩阵;在不同的场景下使用合适的存储结构,实现对问题的建模
                    \item<7-> 顶点数组+逆邻接表
                \end{itemize}
            \end{itemize}
        \end{block}
    \end{frame}

    \begin{frame}
        \frametitle{图}
        \begin{block}{基本操作}
            \begin{itemize}
                \item<1-> 遍历:深度优先搜索,广度优先搜索
                \item<2-> 最短路径:Dijkstra 算法,Floyd 算法
                \item<3-> 最小生成树:Prim 算法,Kruskal 算法
                \item<4-> 拓扑排序:AOV 网,AOE 网
                \item<5-> 关键路径:AOE 网
            \end{itemize}
        \end{block}
    \end{frame}

    \begin{frame}
        \frametitle{图}
        \begin{block}{7.9}
            假设有向图以邻接表作为存储结构。试基于图的深度优先搜索策略写 一算 法,判断有向图中是否存在由顶点 $V_i$ 至顶点 $V_j(i\neq j)$的路径。
        \end{block}
        \begin{block}{}
            假设有向图以邻接表作为存储结构。试基于图的广度优先搜索策略写 一算 法,判断有向图中是否存在由顶点 $V_i$ 至顶点 $V_j(i\neq j)$的路径。
        \end{block}
    \end{frame}

    \subsection{查找表}
    \begin{frame}
        \frametitle{查找表}
        \begin{block}{基本概念}
            \begin{itemize}
                \item<1-> 逻辑结构相关:
                \begin{itemize}
                    \item<2-> 静态、动态查找表
                    \item<3-> 索引,关键字
                    \item<4-> 平均查找长度
                \end{itemize}
                \item<5-> 存储结构相关:
                \begin{itemize}
                    \item<6-> 顺序表,有序表
                    \item<7-> 二叉排序树,平衡二叉排序树,平衡因子
                    \item<8-> 哈希查找表
                \end{itemize}
            \end{itemize}
        \end{block}
    \end{frame}


    \begin{frame}
        \frametitle{查找表}
        \begin{block}{基本操作}
            \begin{itemize}
                \item<1-> 折半/二分查找
                \item<2-> 分块查找
                \item<3-> 二叉排序树的构造、插入、删除和查找
                \item<4-> 平衡二叉排序树的构造,插入和删除
                \item<5-> 哈希表的构造、插入和查找
                \item<6-> 哈希表的冲突处理
            \end{itemize}
        \end{block}
    \end{frame}

    \begin{frame}
        \frametitle{9.5}
        已知关键字序列$\{10, 25,33,19,06,49,37,76,60\}$, 哈希地址空间为 $0\sim 10$, 哈希函数次 $H(Key) = Key\mod 11$, 求: \begin{enumerate}
            \item 用开放定址线性探测法处理冲突,构造哈希表 $HT1$,分别计算在等概率情况下 $HT1$ 查找成功和查找失败的 $ASL$
            \item 用开放定址二次探测法处理冲突,构造 $HT2$ 查找成功的 $ASL$
            \item 用拉链法解决冲突,构造哈希表 $HT3$ 的 $ASL$
        \end{enumerate}
    \end{frame}


    \begin{frame}
        \frametitle{排序算法复杂度及稳定性}
        \begin{tabular}{c | c  c  c  c  c}
            \toprule
            排序方法 & 最好 & 平均 & 最坏 & 空间复杂度 & 稳定性 \\
            \midrule
            直接插入排序 & $O(n)$ & $O(n^2)$ & $O(n^2)$ & $O(1)$ & 稳定 \\
            希尔排序 & $O(n\log n)$ & $O(n(\log n)^2)$ & $O(n(\log n)^2)$ & $O(1)$ & 不稳定 \\
            直接选择排序 & $O(n^2)$ & $O(n^2)$ & $O(n^2)$ & $O(1)$ & 不稳定 \\
            堆排序 & $O(n\log n)$ & $O(n\log n)$ & $O(n\log n)$ & $O(1)$ & 不稳定 \\
            冒泡排序 & $O(n)$ & $O(n^2)$ & $O(n^2)$ & $O(1)$ & 稳定 \\
            快速排序 & $O(n\log n)$ & $O(n\log n)$ & $O(n^2)$ & $O(\log n)$ & 不稳定 \\
            归并排序 & $O(n\log n)$ & $O(n\log n)$ & $O(n\log n)$ & $O(n)$ & 稳定 \\
            基数排序 & $O(d(n+r))$ & $O(d(n+r))$ & $O(d(n+r))$ & $O(n+r)$ & 稳定 \\
            \bottomrule
        \end{tabular}
    \end{frame}

\end{document}